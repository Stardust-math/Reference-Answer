\clearpage
\hideheaderfooter

\vspace*{-5mm} 
\begin{center}
    \large \heiti 序
\end{center}

%\hspace*{2em}\heiti{声明一~~} \songti {此刷题本(或做题本)\footnote{此刷题本(或做题本)模板来自开源\LaTeX 项目\textbf{ExBook} (\url{https://github.com/ExBook/ExBook})。如果你在利用此模板制作刷题本时遇到问题,
%请关注 \faWeixin \,微信公众号:\underline{\textbf{研小布}},后台回复“ExBook”进入交流群。}只是对原书题目的二次排版,仅供个人学习交流使用,不得用于商业用途。如有侵权,请联系删除。}

%\hspace*{2em}\heiti{声明二~~} \songti {此刷题本(或做题本)只有电子版,无任何纸质版,所有售卖此刷题本纸质版的商家均为盗用,与此刷题本制作人无关,请各位同学注意甄别。}

%\hspace*{2em}\heiti{声明三~~} \songti {制作此刷题本的目的是方便大家在考研备考中多次刷题、记录自己的刷题过程和笔迹,以便日后复盘与巩固!
%此刷题本不包含答案,答案请参考原书!\textbf{若在做题中遇到错误,可以点击封面或此处的\underline{\href{\onlinecheckurl}{在线勘误文档}},进行查错和报错}},如链接失效,请关注微信公众号:\underline{\textbf{研小布}},后台回复“勘误文档”获取最新的勘误文档。

% \hspace*{2em}此刷题本模板来自开源项目\textbf{ExBook} (\url{https://github.com/ExBook/ExBook})。如果你在利用此模板制作刷题本时遇到问题,
% 请关注 \faWeixin \,微信公众号:\underline{\textbf{研小布}}(可使用微信扫描下面的二维码关注),后台回复“ExBook”进入交流群。


% \begin{minipage}[t]{1.0\textwidth}
%     \centering
%     \includegraphics[width=0.30\textwidth]{img/qr01.jpg} 
% \end{minipage}

了解到大家貌似没有这本书的参考答案, 为了提高大家的学习效率, 特意制作了这份参考答案, 希望能对大家有所帮助(,,・ω・,,). 我尽量让每道题都保持和原书一致, 不过也有个别字符存在微调, 以便阅读通畅. 因为电子版的图有点糊, 所以有些简单的图表可能我就自己做了, 但是如果有些图实在难做可能就只能从电子版截图放上去了, 希望大家见谅. 以及由于课程只上到第九章, 所以第十章的习题还没有解答, 如果有同学能提供答案的话欢迎补充.

在此感谢\underline{\hbox to 40mm{}}对这份答案的检查校对.

\hspace*{2em}\heiti{注 \textbf{1}~~} \kaishu {如果发现解答有问题或有其他建议, 欢迎发邮件至chenjinghao@mail.ustc.edu.cn或stardust.math26@gmail.com.}

\hspace*{2em}\heiti{注 \textbf{2}~~} \kaishu ``{\textcolor{themeColor}{\selectfont \ding{226}}}''的后面是我个人对于该题相关的知识点的整理或者理解, 一般除了基本定义以外的性质都会加上这些注释, 方便对课本或者相关知识不熟的同学进行学习. 可能有些地方不够严谨, 欢迎大家批评指正. (不出意外的话我应该还会在担任这门课助教前或者期间,根据本班进度做一份讲义,可能也会解释部分习题中的内容.)

\hspace*{2em}\heiti{注 \textbf{3}~~} \kaishu {封面用的是Wallpaper中的"山景", 模板改编自\url{https://www.latexstudio.net/index/details/index/mid/4438.html}。如果有侵权,请联系我删除(╥$_{\sim\sim}$╥).}
\clearpage